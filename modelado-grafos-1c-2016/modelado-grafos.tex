\documentclass[compress]{beamer}

\mode<presentation>
{
  %\usetheme{Warsaw}
  %\usecolortheme{spruce}
  % or ...
	%\useoutertheme{infolines}
  %\setbeamercovered{transparent}
  
  \usetheme{CambridgeUS}
    \setbeamercolor{item projected}{bg=darkred}
    \setbeamertemplate{enumerate items}[default]
    \setbeamertemplate{navigation symbols}{}
    \setbeamercovered{invisible}
    \setbeamercolor{block title}{fg=darkred}
    \setbeamercolor{local structure}{fg=darkred}
  
  % or whatever (possibly just delete it)
}

\usepackage{verbatim} 
\usepackage{listings}
\usepackage{tikz}
\usetikzlibrary{arrows}
\usetikzlibrary{shapes}
\tikzstyle{block}=[draw opacity=0.7,line width=1.4cm]

\newcommand{\bigpause}{\bigskip \pause}

\lstloadlanguages{C++}
\lstnewenvironment{code}
	{%\lstset{	numbers=none, frame=lines, basicstyle=\small\ttfamily, }%
	 \csname lst@SetFirstLabel\endcsname}
	{\csname lst@SaveFirstLabel\endcsname}
\lstset{% general command to set parameter(s)
	language=C++, basicstyle=\footnotesize\sffamily, keywordstyle=\slshape,
	emph=[1]{tipo,usa}, emphstyle={[1]\sffamily\bfseries},
	basewidth={0.47em,0.40em},
	columns=fixed, fontadjust, resetmargins, xrightmargin=5pt, xleftmargin=15pt,
	flexiblecolumns=false, tabsize=2, breaklines,	breakatwhitespace=false, extendedchars=true,
	numbers=left, numberstyle=\tiny, stepnumber=1, numbersep=9pt,
	frame=l, framesep=3pt,
}

\usepackage[spanish]{babel}
% or whatever

\usepackage[utf8]{inputenc}
% or whatever

\usepackage{times}
\usepackage[T1]{fontenc}
% Or whatever. Note that the encoding and the font should match. If T1
% does not look nice, try deleting the line with the fontenc.


\title[Modelado con grafos] % (optional, use only with long paper titles)
{Modelado con grafos}

\author[Melanie Sclar] % (optional, use only with lots of authors)
{~Melanie Sclar}
% - Give the names in the same order as the appear in the paper.
% - Use the \inst{?} command only if the authors have different
%   affiliation.
\institute[UBA] % (optional, but mostly needed)
{
  %\inst{1}%
  Facultad de Ciencias Exactas y Naturales\\
  Universidad de Buenos Aires
}
%\date[TC 2014] % (optional, should be abbreviation of conference name)
%{Training Camp 2014}

% Ac¿ se puede insertar el logo de la UBA
% \pgfdeclareimage[height=0.5cm]{university-logo}{university-logo-filename}
% \logo{\pgfuseimage{university-logo}}



% Delete this, if you do not want the table of contents to pop up at
% the beginning of each subsection:
\AtBeginSubsection[]
{
  \begin{frame}<beamer>{Contenidos}
    \tableofcontents[currentsection,currentsubsection]
  \end{frame}
}

\newcommand{\be}{\begin{equation*}}
\newcommand{\ee}{\end{equation*}}
\newcommand{\state}[1]{\left|\,#1\,\right\rangle}
\newcommand{\costate}[1]{\left\langle\,#1\,\right|}
\newcommand{\trace}{\text{Tr}}
\newcommand{\su}{\uparrow}
\newcommand{\sd}{\downarrow}
\newcommand{\im}{\text{Im}}
\newcommand{\re}{\text{Re}}

% If you wish to uncover everything in a step-wise fashion, uncomment
% the following command:

%\beamerdefaultoverlayspecification{<+->}


\begin{document}
\pgfdeclarelayer{background}
\pgfsetlayers{background,main}
\begin{frame}
  \titlepage
\end{frame}

\begin{frame}{Contenidos}
  \tableofcontents
  % You might wish to add the option [pausesections]
\end{frame}

\section{Cómo recorrer un grafo}
\subsection{Depth First Search}
\begin{frame}{Repaso: DFS}
El DFS (Depth First Search) es un tipo de recorrido del grafo. Se dice que lo recorre \textit{en profundidad}, es decir, empieza por el nodo inicial y en cada paso visita un nodo vecino no visitado del nodo donde está parado, si no hay nodos por visitar vuelve para atrás.

\bigskip
Veamos un ejemplo visual de cómo se recorre un grafo con DFS.
\end{frame}


\begin{frame}{Ejemplo}
\tikzstyle{vertex}=[circle,fill=black!25,minimum size=20pt,inner sep=0pt]
\tikzstyle{selected vertex} = [vertex, fill=red!24]
\tikzstyle{edge} = [draw,thick,-]
\tikzstyle{weight} = [font=\small]
\tikzstyle{selected edge} = [draw,line width=5pt,-,red!50]
\tikzstyle{ignored edge} = [draw,line width=5pt,-,black!20]

\begin{figure}
\begin{tikzpicture}[scale=1.8, auto,swap]
    % Draw a 7,11 network
    % First we draw the vertices
    \foreach \pos/\name/\dist in {{(0,2)/0/0}, {(2,1)/1}, {(4,1)/2}, {(0,0)/3}, {(3,0)/4}, {(2,-1)/5}, {(4,-1)/6}}
        \node[vertex] (\name) at \pos {$\name$};
    % Connect vertices with edges and draw weights
    %\foreach \source/ \dest /\weight in {b/a/7, c/b/8,d/a/5,d/b/9,
           %                              e/b/7, e/c/5,e/d/15,
          %                               f/d/6,f/e/8,
         %                                g/e/9,g/f/11}
        %\path[edge] (\source) -- node[weight] {$\weight$} (\dest);
      \path[edge] (1) -- (0);
      \path[edge] (2) -- (1);
      \path[edge] (3) -- (0);
      \path[edge] (4) -- (3);
      \path[edge] (5) -- (3);
      \path[edge] (5) -- (4);
      \path[edge] (6) -- (5);
      
    \foreach \vertex / \fr in {0/1,1/2,2/3,3/4,5/5,4/6,6/7}
        \path<\fr-> node[selected vertex] at (\vertex) {$\vertex$};
    % For convenience we use a background layer to highlight edges
    % This way we don't have to worry about the highlighting covering
    % weight labels. 
    \begin{pgfonlayer}{background}
        \pause
        \foreach \source / \dest / \fr in {0/1/2,1/2/3,0/3/4,3/5/5,5/4/6,5/6/7}
            \path<\fr->[selected edge] (\source.center) -- (\dest.center);
    \end{pgfonlayer}
\end{tikzpicture}
\end{figure}
\end{frame}


\begin{frame}{Ejemplo}
En el ejemplo anterior vimos cómo recorre el grafo un DFS. Ahora tratemos de implementar un DFS resolviendo un problema (por supuesto, DFS se puede aplicar a muchos problemas más).

\begin{block}{Problema}
Dado un grafo conexo (es decir, que existe al menos un camino entre todo par de nodos) queremos ver si dicho grafo es un {\bf árbol}.
\end{block}
\end{frame}

\begin{frame}{Ideas para la resolución}
\begin{itemize}
\item Primero que nada, como tenemos un grafo vamos a querer recorrerlo y detectar si existe más de un camino entre algún par de nodos. ¿Pero cómo podemos hacer esto?
\pause
\invisible<1>{
\item Iremos recorriendo el grafo con DFS, siempre moviéndome a un vecino no visitado aún. Para no visitar un nodo dos veces (eso sería innecesario y empeoraría la performance) marcamos cada nodo cuando lo revisamos.  ¿Qué significa si llego a un nodo y un vecino de él (que no es por el que llegué) ya está visitado?}
\pause
\invisible<1-2>{
\item Significa que desde mi nodo inicial pude llegar a un nodo de dos formas distintas: es decir que no existe un único camino, lo que contradice la definición de árbol. Ergo, no es árbol. }
\end{itemize}
\end{frame}


%\pause
%\invisible<1>{
%\item Cuando llega a un nodo que ya fue visitado se da cuenta de que pudo acceder a ese nodo por dos caminos, eso %quiere decir que si recorremos uno de los caminos en un sentido y el otro camino en sentido opuesto formamos un ciclo.}
%\pause
%\invisible<1-2>{
%\item Asx como el BFS se implementa con una cola, el DFS se implementa con una pila.}
%\end{itemize}
%\end{frame}

\begin{frame}[fragile]{Pseudocódigo del DFS}
\begin{lstlisting}
// inicialmente se llena el vector visitado en false pues ningun nodo fue visitado
// ademas el nodo inicial no tiene padre y por eso se llama con padre = -1
bool esArbol(grafo, int nodoActual, vector<bool> &visitado, int padre):
    visitado[nodoActual] = true 

    para cada vecino v de nodoActual en grafo:
        si v ya fue visitado y v != padre:
            devolver false
            // encontre dos caminos distintos y luego no es arbol
        si v no fue visitado aun:
            si no esArbol(grafo, v, visitado, nodoActual): 
            // llamo recursivamente al dfs con el subarbol con raiz en v, ahora el padre es nodoActual
                devolver false
    
    // si para ningun vecino encontre algun ciclo, es un arbol porque se que es conexo
    devolver true
		
\end{lstlisting}
\end{frame}

\begin{frame}[fragile]{Implementación en C++ del DFS}
\begin{lstlisting}
bool esArbol(vector<vector<int> > &lista, int t, vector<bool> &visitado, int padre)
{
    visitado[t] = true;
    for(int i = 0; i < lista[t].size(); i++)
    {
        if(visitado[lista[t][i]] == true && lista[t][i] != padre)
            return false;
        if(visitado[lista[t][i]] == false)
            if(esArbol(lista, lista[t][i], visitado, t) == false)
                return false;
    }
    return true;
}
\end{lstlisting}
\end{frame}

\begin{frame}{Análisis del DFS}
\begin{itemize}
\item En ningún momento usamos una pila explícita, pero la pila está implícita en la recursión.
\pause
\invisible<1>{
\item Guardamos el padre del nodo, es decir, el nodo desde el cual fuimos a parar al nodo actual, para no confundir un ciclo con ir y volver por la misma arista.}
\pause
\invisible<1-2>{
\item Si el nodo ya lo visitamos y no es el nodo desde el cual venimos quiere decir que desde algún nodo llegamos por dos caminos, o sea que hay un ciclo.}
\pause
\invisible<1-3>{
\item Si el nodo no lo visitamos, pero desde uno de sus vecinos podemos llegar a un ciclo, entonces es porque hay un ciclo en el grafo y por lo tanto no es un árbol.}
\end{itemize}
\end{frame}

\subsection{Breadth First Search}
\begin{frame}{Otro algoritmo de búsqueda: BFS}
Muchas veces no basta con recorrer el grafo, sino que queremos hacerlo de una forma en particular. BFS suele ser muy útil en muchos problemas, veamos de qué se trata!
\bigskip

\begin{block}{Breadth First Search o \textit{búsqueda en anchura}}
El Breadth First Search o \textit{búsqueda en anchura} es un algoritmo que recorre un grafo comenzando por un nodo en particular, y se exploran todos los vecinos del nodo. Luego, se exploran todos los vecinos de los vecinos del nodo inicial, y así siguiendo hasta recorrer todos los nodos.
\end{block}

Veámoslo gráficamente.
\end{frame}

\begin{frame}{Ejemplo de recorrido de un grafo usando BFS}
\tikzstyle{vertex}=[circle,fill=black!25,minimum size=20pt,inner sep=0pt]
\tikzstyle{selected vertex} = [vertex, fill=red!24]
\tikzstyle{edge} = [draw,thick,-]
\tikzstyle{weight} = [font=\small]
\tikzstyle{selected edge} = [draw,line width=5pt,-,red!50]
\tikzstyle{ignored edge} = [draw,line width=5pt,-,black!20]

\begin{figure}
\begin{tikzpicture}[scale=1.8, auto,swap]
    % Draw a 7,11 network
    % First we draw the vertices
    \foreach \pos/\name/\dist in {{(0,2)/0/0}, {(2,1)/1}, {(4,1)/2}, {(0,0)/3}, {(3,0)/4}, {(2,-1)/5}, {(4,-1)/6}}
        \node[vertex] (\name) at \pos {$\name$};
    % Connect vertices with edges and draw weights
    %\foreach \source/ \dest /\weight in {b/a/7, c/b/8,d/a/5,d/b/9,
           %                              e/b/7, e/c/5,e/d/15,
          %                               f/d/6,f/e/8,
         %                                g/e/9,g/f/11}
        %\path[edge] (\source) -- node[weight] {$\weight$} (\dest);
      \path[edge] (1) -- (0);
      \path[edge] (2) -- (1);
      \path[edge] (3) -- (0);
      \path[edge] (4) -- (3);
      \path[edge] (5) -- (3);
      \path[edge] (5) -- (4);
      \path[edge] (6) -- (5);
      
    \foreach \vertex / \fr in {0/1,1/2,3/3,2/4,4/5,5/6,6/7}
        \path<\fr-> node[selected vertex] at (\vertex) {$\vertex$};
    % For convenience we use a background layer to highlight edges
    % This way we don't have to worry about the highlighting covering
    % weight labels. 
    \begin{pgfonlayer}{background}
        \pause
        \foreach \source / \dest / \fr in {0/1/2,0/3/3,1/2/4,3/4/5,3/5/6,5/6/7}
            \path<\fr->[selected edge] (\source.center) -- (\dest.center);
    \end{pgfonlayer}
\end{tikzpicture}
\end{figure}
\end{frame}


\begin{frame}
Implementemos un BFS resolviendo un problema. Vale notar que este problema no podía ser resuelto con DFS.

\begin{block}{Calcular las distancias de un nodo a todos los demás}
Dado un nodo inicial, queremos hallar la distancia de cada nodo a todos los demás. Recordemos que puede haber más de una forma de ir de un nodo a otro, pero para la distancia siempre tomaremos la mínima.
\end{block}

\textquestiondown Cómo podemos resolverlo usando BFS? Por ahora no nos centremos en cómo se implementa el algoritmo, sino en cómo usarlo para resolver el problema. Luego intentaremos implementarlo.
\end{frame}

\begin{frame}{Detalles de implementación del BFS}
\begin{itemize}
\item Las distancias del nodo inicial a los demás nodos las inicializaremos en infinito (en principio no sabemos a qué distancia están, luego iremos actualizando el valor si encontramos un camino desde el nodo inicial al nodo).
\pause
\invisible<1>{
\item Usaremos una cola para ir agregando los nodos por visitar. Como agregamos primero todos los vecinos del nodo inicial, los primeros nodos en entrar a la cola son los de distancia 1, luego agregamos los vecinos de esos nodos, que son los de distancia 2, y así vamos recorriendo el grafo en orden de distancia al vértice inicial.}
\pause
\invisible<1-2>{
\item Cuando visitamos un nodo, sabemos cuáles de sus vecinos agregar a la cola. Tenemos que visitar los vecinos que todavía no han sido visitados.}
\end{itemize}
\end{frame}

\begin{frame}[fragile]{Pseudoc\'odigo del BFS}

\begin{lstlisting}
vector<int> BFS (Grafo listaAdyacencias, int nodoInicial):
   cantidadDeNodos = longitud(listaAdyacencias)
   queue<int> cola // aqui encolare los nodos que todavia no analizamos

   // distancias[i] = distancia de i a nodoInicial, inicialmente es INFINITO
   vector<int> distancias(cantidadDeNodos, INFINITO) 

   cola.encolar(nodoInicial)
   distancias[nodoInicial] = 0
   mientras (la cola no esta vacia):
       tope = cola.tope() // tomo el tope de la cola (y lo saco de la cola)

       para todos los vecinos v de "tope" en el grafo:
           si la distancia entre v y el nodoInicial es infinito:
               distancias[v] = distancias[tope] + 1 // esta un nodo mas lejos que su vecino
               cola.encolar(v) // encolo v para analizarlo luego

   devolver distancias
\end{lstlisting}
\end{frame}


\begin{frame}[fragile]{Implementaci\'on del BFS}
\begin{lstlisting}
vector<int> BFS(vector<vector<int> > &lista, int nodoInicial){
    int n = lista.size(),t;
    queue<int> cola;
    vector<int> distancias(n,n);
    cola.push(nodoInicial);
    distancias[nodoInicial] = 0;
    while(!cola.empty()){
        t = cola.front();
        cola.pop();
        for(int i=0;i<lista[t].size();i++){
            if(distancias[lista[t][i]]==n){
                distancias[lista[t][i]] = distancias[t]+1;
                cola.push(lista[t][i]);
            }
        }
    }
    return distancias;
}
\end{lstlisting}
\end{frame}

\section{Problemas con grafos}
\subsection{Calentando motores}
\begin{frame}{Problema sencillo para empezar}
En una ciudad con forma de grilla hay $h$ calles horizontales y $v$ calles verticales. Adem\'as, algunas esquinas est\'an obstaculizadas por diversos motivos, y no se puede pasar por all\'i.\bigskip

Queremos partir desde nuestra casa y llegar a la facultad. ?`Ser\'a posible lograrlo? \\ Dise\~nemos un algoritmo eficiente para resolver este problema. \bigskip

{\bf Pista:} el problema se puede resolver con complejidad temporal $O(v h)$

\end{frame}

\subsection{Modelar un problema con un grafo no trivial}
\begin{frame}{Modelar un problema con un grafo no trivial}
Muchas veces, modelar un problema con el grafo que el enunciado casi expl\'icitamente nos describe no es suficiente para cumplir con los requerimientos de eficiencia que se nos piden (o a veces, ni siquiera sabr\'iamos c\'omo resolverlo con ese grafo que nos dan).\bigskip

Para eso, durante la clase de hoy intentaremos {\bf pensar en grafos un poco m\'as ricos de los que venimos viendo, con m\'as par\'ametros}, que nos permitan resolver los problemas eficientemente. \bigskip

Como suele suceder, esto es un trade-off: {\bf al ganar en complejidad temporal, podemos perder en complejidad espacial}.
\end{frame}

\subsection{En la altura, la bici no dobla}
\begin{frame}{En la altura, la bici no dobla}
Melanie vive en la ciudad que mencionamos anteriormente: tiene forma de grilla, con $h$ calles horizontales y $v$ calles verticales. Adem\'as, algunas esquinas est\'an obstaculizadas por diversos motivos, y no puede pasar por all\'i.\bigskip

Recientemente Melanie se compr\'o una bicicleta y ya aprendi\'o a andar sin rueditas, pero doblar todav\'ia le cuesta. Le gustar\'ia poder ir en bici de su casa a la facultad minimizando la cantidad de veces que tiene que doblar en una esquina.
?`C\'omo la podemos ayudar? \bigskip
\end{frame}

\begin{frame}{Spoilers - idea 1}

Crear un grafo donde cada nodo describa una esquina y la direcci\'on de la que provenimos con la bici. As\'i, {\bf el nodo ser\'a una tupla (fila, columna, direcci\'on)}, y se conectar\'an dos nodos si y s\'olo si de un estado puedo pasar al otro. Adem\'as, si para pasar de un estado a otro debo doblar en una esquina, la arista tendr\'a costo 1. De lo contrario, tendr\'a costo 0.
\bigskip

Luego, tendremos un grafo con aristas de peso 0 \'o 1, y queremos saber cu\'al es el camino de costo m\'inimo desde la esquina inicial hasta la final. O(vh lg(vh)) si usamos Dijkstra con cola de prioridad.

\end{frame}


\begin{frame}{Spoilers - idea 2}

Ahora cada nodo adem\'as contendr\'a cu\'antos giros fue necesario realizar para llegar a cada esquina. As\'i, {\bf el nodo ser\'a una tupla (fila, columna, cantidad de giros usados hasta el momento, direcci\'on)}, y se conectar\'an dos nodos si y s\'olo si de un estado puedo pasar al otro.
\bigskip

Notar que este grafo no tendr\'a pesos, y lo que queremos preguntar es si ser\'a posible llegar a alg\'un nodo final cualquiera sea la cantidad de giros y la direcci\'on del mismo. Para ello ser\'a necesario recorrer el grafo (por ejemplo usando BFS) y ver si el nodo salida y alguna de las llegadas se encuentran en la misma componente conexa. Linealmente recorremos todos los posibles nodos llegada, y retornamos aquel cuya cantidad de giros necesarios sea m\'inima. \bigskip

\end{frame}

\begin{frame}{Spoilers - idea 2 (c\'alculo de complejidad)}

Llamemos $g$ a la cantidad m\'axima de giros posibles a realizar en el tablero.
As\'i, como BFS es lineal en el tama\~no de la entrada, y tenemos $v \times h \times g \times 2$ nodos en nuestro modelado (y cada nodo tiene grado a lo sumo 4), la recorrida del grafo ser\'a O(vhg).
\bigskip

Al final recorremos linealmente todos los posibles nodos finales, que como tienen fija la esquina en la que terminan son solamente $2g$ nodos. As\'i, la complejidad total del algoritmo es $O(vhg) + O(2g) = O(vhg)$
\end{frame}


\subsection{El retorno del jedi recargado}
\begin{frame}{El retorno del jedi recargado}
Luke también vive en la ciudad que mencionamos anteriormente. Quiere ir desde la esquina superior izquierda hasta la esquina inferior derecha, y como no quiere que le digan \textit{que corre para atrás porque no tiene aguante}, va a intentar sólo moverse hacia la derecha o hacia abajo.\\ \bigskip

Sin embargo, no es tan tajante como en el TP: sabe que puede ser imposible llegar de una punta a la otra con solo estos dos movimientos pues hay obstáculos. Entonces, está dispuesto a moverse hasta $a \geq 0$ veces hacia la izquierda y $b \geq 0$ veces hacia arriba. Puede moverse para abajo o para la derecha tantas veces como quiera.\\ \bigskip

¿Dadas estas restricciones, cuál es la mínima cantidad de calles que tiene que recorrer Luke para llegar a su destino?
\end{frame}

\begin{frame}
\begin{center}
{¡A pensar!}
\end{center}
\end{frame}

\end{document}