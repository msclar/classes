\documentclass[compress]{beamer}

\mode<presentation>
{
  %\usetheme{Warsaw}
  %\usecolortheme{spruce}
  % or ...
	%\useoutertheme{infolines}
  %\setbeamercovered{transparent}
  
  \usetheme{CambridgeUS}
    \setbeamercolor{item projected}{bg=darkred}
    \setbeamertemplate{enumerate items}[default]
    \setbeamertemplate{navigation symbols}{}
    \setbeamercovered{transparent}
    \setbeamercolor{block title}{fg=darkred}
    \setbeamercolor{local structure}{fg=darkred}
  
  % or whatever (possibly just delete it)
}

\usepackage{verbatim} 
\usepackage{listings}
\usepackage{tikz}
\usetikzlibrary{arrows}
\usetikzlibrary{shapes}
\tikzstyle{block}=[draw opacity=0.7,line width=1.4cm]

\newcommand{\bigpause}{\bigskip \pause}

\lstloadlanguages{C++}
\lstnewenvironment{code}
	{%\lstset{	numbers=none, frame=lines, basicstyle=\small\ttfamily, }%
	 \csname lst@SetFirstLabel\endcsname}
	{\csname lst@SaveFirstLabel\endcsname}
\lstset{% general command to set parameter(s)
	language=C++, basicstyle=\footnotesize\sffamily, keywordstyle=\slshape,
	emph=[1]{tipo,usa}, emphstyle={[1]\sffamily\bfseries},
	basewidth={0.47em,0.40em},
	columns=fixed, fontadjust, resetmargins, xrightmargin=5pt, xleftmargin=15pt,
	flexiblecolumns=false, tabsize=2, breaklines,	breakatwhitespace=false, extendedchars=true,
	numbers=left, numberstyle=\tiny, stepnumber=1, numbersep=9pt,
	frame=l, framesep=3pt,
}

\usepackage[spanish]{babel}
% or whatever

\usepackage[utf8]{inputenc}
% or whatever

\usepackage{times}
\usepackage[T1]{fontenc}
% Or whatever. Note that the encoding and the font should match. If T1
% does not look nice, try deleting the line with the fontenc.


\title[Geometr\'ia Computacional] % (optional, use only with long paper titles)
{Geometr\'ia Computacional}

\author[Melanie Sclar] % (optional, use only with lots of authors)
{~Melanie Sclar}
% - Give the names in the same order as the appear in the paper.
% - Use the \inst{?} command only if the authors have different
%   affiliation.
\institute[UBA] % (optional, but mostly needed)
{
  %\inst{1}%
  Facultad de Ciencias Exactas y Naturales\\
  Universidad de Buenos Aires
}
\date[PAP] % (optional, should be abbreviation of conference name)
{Problemas, Algoritmos y Programación}

% Ac¿ se puede insertar el logo de la UBA
% \pgfdeclareimage[height=0.5cm]{university-logo}{university-logo-filename}
% \logo{\pgfuseimage{university-logo}}



% Delete this, if you do not want the table of contents to pop up at
% the beginning of each subsection:
\AtBeginSubsection[]
{
  \begin{frame}<beamer>{Contenidos}
    \tableofcontents[currentsection,currentsubsection]
  \end{frame}
}

\newcommand{\be}{\begin{equation*}}
\newcommand{\ee}{\end{equation*}}
\newcommand{\state}[1]{\left|\,#1\,\right\rangle}
\newcommand{\costate}[1]{\left\langle\,#1\,\right|}
\newcommand{\trace}{\text{Tr}}
\newcommand{\su}{\uparrow}
\newcommand{\sd}{\downarrow}
\newcommand{\im}{\text{Im}}
\newcommand{\re}{\text{Re}}

% If you wish to uncover everything in a step-wise fashion, uncomment
% the following command:

%\beamerdefaultoverlayspecification{<+->}


\begin{document}
\pgfdeclarelayer{background}
\pgfsetlayers{background,main}
\begin{frame}
  \titlepage
\end{frame}

\begin{frame}{Contenidos}
  \tableofcontents
  % You might wish to add the option [pausesections]
\end{frame}

\section{Técnicas de barrido}
\subsection{Sweep circle}

\begin{frame}{¿Qu\'e es sweep circle?}
\begin{itemize}
\item La idea de sweep circle es exactamente la misma que la ya mencionada sweep line, pero en lugar de mover una recta imaginaria, movemos una circunferencia.

\item La circunferencia puede moverse en linea recta (traslación) o alrededor de un centro fijo (rotación).

\item Como un círculo es una figura acotada, el choque de la circunferencia con los puntos interesantes produce eventos de \textit{entrada} y de \textit{salida} en el círculo.
\end{itemize}
\end{frame}

\begin{frame}{Ejemplo: Ubicación ideal de un círculo en el eje Y}

\begin{block}{Problema}
    Dado un radio $R > 0$ entero, se debe indicar cuál es la máxima cantidad de puntos de la grilla de coordenadas enteras que es posible encerrar con un círculo de radio $R$, cuyo centro se encuentre posicionado sobre la recta $x=0$ (el eje $y$).
\end{block}

\pause
\invisible<1-1>
{
    Observación: Alcanza con considerar las posiciones $0 \leq y \leq 1$
}

\end{frame}

\begin{frame}{Planteo con sweep circle}

\begin{itemize}
    \item Comenzamos con el círculo ubicado en $(0,0)$, y todos los correspondientes puntos de la grilla adentro.
    \item ``Movemos'' el círculo en vertical, hasta llegar a $(0,1)$, procesando los eventos de entrada y salida de puntos.
    \item La máxima cantidad de puntos que tengamos dentro del círculo en cualquier momento, es el resultado.
\end{itemize}

\pause

\invisible<1-1>{
    \begin{itemize}
       \item Notar que hay solamente $O(R)$ eventos de entrada / salida, y además la cantidad de puntos totales dentro del círculo inicial puede computarse en $O(R)$.
       \item Con todo esto y la técnica de barrido, el problema se resuelve en $O(R \lg R)$
    \end{itemize}
}

\end{frame}

\begin{frame}{Problemas para pensar}

\begin{itemize}
\item goo.gl/rT7Ji
\item goo.gl/llEHC
\end{itemize}

\end{frame}

\subsection{Sweep line - problemas más complejos}
\begin{frame}{Área de unión de rectángulos}
%%%%%%%%%%% TODO %%%%%%%%%%%%%

\end{frame}


\section{Planaridad (SECCION DE AGUS SIN REVISAR)}

\subsection{Definiciones}

\begin{frame}{Grafo planar}

\begin{block}{Definición}
    Un grafo se dice \textit{planar} si es posible dibujarlo en el plano, haciendo corresponder a cada vértice un punto, y a cada arista una curva simple continua que una los puntos correspondientes a los extremos de la arista, de manera tal que dos curvas correspondientes a aristas distintas no se intersequen más que en sus extremos.
\end{block}

\begin{block}{Definición}
    Dado un grafo planar $G$, a un dibujo de $G$ en el plano que cumple lo enunciado en la definición anterior se lo denomina un \textit{embedding}, \textit{inmersión} o simplemente \textit{dibujo} de $G$.
\end{block}

Notar que un mismo grafo planar $G$ puede tener infinitos embeddings distintos.

\end{frame}

\begin{frame}{Región}

\begin{block}{Definición}
    Dado un embedding $E$ de un grafo planar $G$, se denomina una \textit{región} de $E$ a una componente conexa del conjunto de puntos del plano que no forman parte del dibujo de $G$ en $E$.
\end{block}

Notar que al igual que muchas otras propiedades de un dibujo de un grafo planar, el conjunto de regiones depende del dibujo, y \textbf{en principio}, distintos dibujos de un mismo grafo planar podrían tener diferente cantidad de regiones.

\end{frame}

\begin{frame}{Región (cont)}

\begin{block}{Definición}
    Dada una región $f$ de un dibujo de un grafo planar $G$, se denomina el \textit{grado} de $f$ y lo notaremos $d(f)$, a la cantidad de aristas presentes en la \textit{frontera} de $f$ en el dibujo. Además, si la región $f$ toca
    a la arista de ambos lados, entonces será contada dos veces para el grado.
\end{block}

Notar que de la definición surge que cada arista ``aporta grado'' a exactamente dos regiones (o bien, a una misma región dos veces), de donde siempre se tiene $\sum_f {d(f)} = 2m$

\end{frame}

\subsection{Fórmula de Euler}

\begin{frame}{Fórmula de Euler}

La principal herramienta para trabajar con grafos planares es el siguiente resultado:

\begin{block}{Teorema}
    Si $G$ es un grafo planar conexo de $n$ vértices y $m$ aristas, y $R$ es la cantidad de regiones de \textbf{cualquier} dibujo de $G$, entonces:    
    $$R + n = m + 2 \mbox{ (fórmula de Euler)}$$
    
    En general, para un grafo planar cualquiera con $c \geq 1$ componentes conexas vale:    
    $$R + n = m + c + 1$$
\end{block}

Observar que esto es válido incluso si el grafo contiene multiejes (más de un eje entre un mismo par de nodos) y bucles (ejes de un nodo a sí mismo).

\end{frame}

\begin{frame}{Raleza de los grafos planares}

Sea $G$ un grafo simple (sin multiejes ni bucles) planar, y $g$ la longitud mínima de un ciclo simple de $g$ (si $G$ no tiene ciclos tendremos directamente $m \leq n-1$).

\begin{block}{Teorema}
    Si $G$ cumple lo anterior, entonces $m \leq \frac{(n-c-1)g}{g-2}$.
\end{block}
\begin{block}{Corolario}
    Si $G$ es grafo simple planar con $n \geq 3$, entonces $m \leq 3n - 6$.
\end{block}

Para demostrar esto, notamos que la frontera de una región debe contener un circuito, así que
$$2m = \sum_f {d(f)} \geq R g = (m-n+c+1) g \Rightarrow m \leq \frac{(n-c-1)g}{g-2}$$

\end{frame}



\begin{frame}{Ejemplos mínimos de grafos no planares}

Como consecuencia de lo anterior, notamos que:

\begin{itemize}
    \item $K_5$ no es planar: tiene $m=10$ y $n=5$, y no cumple $m \leq 3n-6$.
    \item $K_{3,3}$ no es planar: tiene $m=9$ , $n=6$ , $g=4$ y $c=1$, y no cumple $m \leq \frac{(n-c-1)g}{g-2} = \frac{(6-1-1)4}{2} = 8$.
\end{itemize}

Estos son los ejemplos no planares con menor cantidad de nodos y aristas, respectivamente.

\end{frame}

\begin{frame}{Referencias}
   \begin{itemize}
   \item \textit{Introduction to Algorithms, 2nd Edition}. MIT Press. \\ Thomas H. Cormen
   \textbf{Sección 33} (Computational Geometry)
   \item \url{https://www.topcoder.com/tc?module=Static&d1=tutorials&d2=geometry1}
   \item \url{https://www.topcoder.com/tc?module=Static&d1=tutorials&d2=geometry2}
   \item \url{https://www.topcoder.com/tc?module=Static&d1=tutorials&d2=geometry3}
   \item \url{https://www.topcoder.com/tc?module=Static&d1=tutorials&d2=lineSweep}
  \end{itemize}
  
\end{frame}


\end{document}
