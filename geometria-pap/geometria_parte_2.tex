\documentclass[compress]{beamer}

\mode<presentation>
{
  %\usetheme{Warsaw}
  %\usecolortheme{spruce}
  % or ...
	%\useoutertheme{infolines}
  %\setbeamercovered{transparent}
  
  \usetheme{CambridgeUS}
    \setbeamercolor{item projected}{bg=darkred}
    \setbeamertemplate{enumerate items}[default]
    \setbeamertemplate{navigation symbols}{}
    \setbeamercovered{transparent}
    \setbeamercolor{block title}{fg=darkred}
    \setbeamercolor{local structure}{fg=darkred}
  
  % or whatever (possibly just delete it)
}

\usepackage{verbatim} 
\usepackage{listings}
\usepackage{tikz}
\usetikzlibrary{arrows}
\usetikzlibrary{shapes}
\tikzstyle{block}=[draw opacity=0.7,line width=1.4cm]

\newcommand{\bigpause}{\bigskip \pause}

\lstloadlanguages{C++}
\lstnewenvironment{code}
	{%\lstset{	numbers=none, frame=lines, basicstyle=\small\ttfamily, }%
	 \csname lst@SetFirstLabel\endcsname}
	{\csname lst@SaveFirstLabel\endcsname}
\lstset{% general command to set parameter(s)
	language=C++, basicstyle=\footnotesize\sffamily, keywordstyle=\slshape,
	emph=[1]{tipo,usa}, emphstyle={[1]\sffamily\bfseries},
	basewidth={0.47em,0.40em},
	columns=fixed, fontadjust, resetmargins, xrightmargin=5pt, xleftmargin=15pt,
	flexiblecolumns=false, tabsize=2, breaklines,	breakatwhitespace=false, extendedchars=true,
	numbers=left, numberstyle=\tiny, stepnumber=1, numbersep=9pt,
	frame=l, framesep=3pt,
}

\usepackage[spanish]{babel}
% or whatever

\usepackage[utf8]{inputenc}
% or whatever

\usepackage{times}
\usepackage[T1]{fontenc}
% Or whatever. Note that the encoding and the font should match. If T1
% does not look nice, try deleting the line with the fontenc.


\title[Geometr\'ia Computacional] % (optional, use only with long paper titles)
{Geometr\'ia Computacional}

\author[Melanie Sclar] % (optional, use only with lots of authors)
{~Melanie Sclar}
% - Give the names in the same order as the appear in the paper.
% - Use the \inst{?} command only if the authors have different
%   affiliation.
\institute[UBA] % (optional, but mostly needed)
{
  %\inst{1}%
  Facultad de Ciencias Exactas y Naturales\\
  Universidad de Buenos Aires
}
\date[PAP] % (optional, should be abbreviation of conference name)
{Problemas, Algoritmos y Programación}

% Ac¿ se puede insertar el logo de la UBA
% \pgfdeclareimage[height=0.5cm]{university-logo}{university-logo-filename}
% \logo{\pgfuseimage{university-logo}}



% Delete this, if you do not want the table of contents to pop up at
% the beginning of each subsection:
\AtBeginSubsection[]
{
  \begin{frame}<beamer>{Contenidos}
    \tableofcontents[currentsection,currentsubsection]
  \end{frame}
}

\newcommand{\be}{\begin{equation*}}
\newcommand{\ee}{\end{equation*}}
\newcommand{\state}[1]{\left|\,#1\,\right\rangle}
\newcommand{\costate}[1]{\left\langle\,#1\,\right|}
\newcommand{\trace}{\text{Tr}}
\newcommand{\su}{\uparrow}
\newcommand{\sd}{\downarrow}
\newcommand{\im}{\text{Im}}
\newcommand{\re}{\text{Re}}

% If you wish to uncover everything in a step-wise fashion, uncomment
% the following command:

%\beamerdefaultoverlayspecification{<+->}


\begin{document}
\pgfdeclarelayer{background}
\pgfsetlayers{background,main}
\begin{frame}
  \titlepage
\end{frame}

\begin{frame}{Contenidos}
  \tableofcontents
  % You might wish to add the option [pausesections]
\end{frame}

\section{T\'ecnicas de barrido}

\subsection{T\'ecnicas de barrido}
\begin{frame}{¿Qu\'e es sweep line?}
\begin{itemize}
\item La t\'ecnica de \emph{sweep line} se caracteriza por tener una l\'inea (vertical u horizontal generalmente, pero no necesariamente) que va barriendo todo el plano y recolectando informaci\'on sobre puntos de \'el.

\bigskip 
%
\item Si se recuerdan ciertos {\it eventos} que van sucediendo a lo largo del barrido y se los analiza, se pueden obtener soluciones de una complejidad mejor que con otras t\'ecnicas.
\bigskip
\item \textbf{Los eventos ser\'an instantes del barrido en los que sucede algo relevante para el problema.}
\end{itemize}
\end{frame}

\begin{frame}
En alg\'un sentido, la t\'ecnica de sweep line es como la programaci\'on din\'amica: uno aprende la idea general, pero luego en cada problema se aplica y se utiliza de manera diferente, adaptando la idea general ya vista.
\bigskip

Es por eso que creemos que la mejor manera de aprender esta t\'ecnica es viendo algunos ejemplos de problemas relevantes en los que se la utiliza.
\end{frame}

\subsubsection{Par de puntos m\'as cercano}

\begin{frame}
\begin{block}{Problema}
Dados $n$ puntos en el plano, queremos encontrar dos tales que la distancia entre ellos sea la m\'inima posible (o sea, el par m\'as cercano)
\end{block}
\bigskip
¿Ideas? \pause 
\invisible<1>{
¿C\'omo s\'e qu\'e idea puede funcionar y cu\'al no, si no s\'e cu\'anto vale $n$? Si $n \leq 1000$, el problema a resolver es muy distinto que si $n \leq 1000000$.
}
\end{frame}

\begin{frame}
\begin{itemize}
\item Existe una soluci\'on trivial en $O(n^2)$, que es simplemente mirar cada par y elegir el de menor distancia entre todos los posibles.
\bigskip

\item Es una soluci\'on muy f\'acil de implementar, as\'i que si el problema es con $n \leq 1000$ definitivamente deber\'iamos programar esa.
\bigskip

\item ¿Pero y si $n\leq 1000000$? Utilizaremos sweep line para encontrar una soluci\'on de mejor complejidad.
\end{itemize}
\end{frame}

\begin{frame}
\begin{itemize}
\item El problema con el enfoque $O(n^2)$ es que a cada punto lo comparamos con \textbf{todos} los dem\'as. 
Tal vez podemos aprovecharnos de la posici\'on en el plano de cada uno para no tener que hacer esto cada vez.
\bigskip
\item Barreremos el plano con una recta vertical, de izquierda a derecha.
\bigskip
\item Para ser coherentes con este barrido, ordenaremos los puntos crecientemente seg\'un su coordenada $x$. 
Esto tendr\'a una complejidad de $O(n \lg n)$.
\end{itemize}
\end{frame}


\begin{frame}
\begin{itemize}
\item Llamemos $h$ a la menor distancia entre dos puntos encontrada hasta el momento. 
La inicializaremos en $\infty$ (en la pr\'actica, un valor tan grande tal que sea imposible que haya un par de puntos tan lejanos).

\bigskip

\item Supongamos que ya pasamos por primeros $i-1$ puntos con la l\'inea vertical, y ahora chocamos al $i$-\'esimo punto.
\end{itemize}
\end{frame}

\begin{frame}

Mantendremos un set con los puntos (ya procesados) cuya coordenada $x$ est\'e a una distancia menor a $h$. 
Estos puntos son los \'unicos candidatos a estar a distancia menor que $h$ del punto $i$-\'esimo, y por ende a ser una mejor soluci\'on que la actual. \bigskip

\begin{center}
\includegraphics[scale=0.6]{images/closest2.png}
\end{center}
% CAMBIAR EL N POR I EN LA IMAGEN

\end{frame}

\begin{frame}
Tambi\'en fij\'emonos que no cualquier punto cuya coordenada $x$ est\'e a distancia menor a $h$ es un candidato real 
a ser una mejor soluci\'on. Si la coordenada $y$ de un punto est\'a a una distancia mayor a $h$, tampoco ser\'a un candidato. 

\begin{center}
\includegraphics[scale=0.6]{images/closest.png}
\end{center}

\end{frame}

\begin{frame}
As\'i, s\'olo deber\'iamos quedarnos con los puntos cuya coordenada $y$ est\'e en el intervalo $(y_i-h,y_i+h)$. \bigskip

¿C\'omo descartamos los puntos con coordenada $x$ mayor a $h$, si tenemos a los candidatos en un set? \bigpause

\invisible<1>{
¡Manteniendo el set ordenado por su coordenada $y$! 

As\'i, \emph{quitar} los puntos cuya coordenada $y$ est\'a demasiado lejos de $p_i$ se reduce simplemente a no mirar los 
puntos fuera de un intervalo.
}
\end{frame}

\begin{frame}

Y ahora, para todos los puntos que quedaron en el set calculamos su distancia con $p_i$ y actualizamos $h$.

¡¿Pero c\'omo?! Potencialmente, hay muchos puntos en el set. \bigpause

\invisible<1>{
Esto no es tan as\'i. Veamos que quedaron a lo sumo 6 candidatos (6 elementos en el set). 
}
\end{frame}

\begin{frame}
La observaci\'on importante es que todos los puntos anteriores est\'an a una distancia mayor o igual a $h$ entre s\'i, 
pues si no, la m\'inima distancia ya habr\'ia sido actualizada anteriormente.

\begin{center}
\includegraphics[scale=0.3]{images/seis_cuadrados.png}
\end{center}

No puede haber 2 puntos en el mismo cuadradito, pues luego esos 2 estar\'ian a una distancia menor que $h$. As\'i, hay a 
lo sumo 6 puntos para mirar.

\end{frame}

\begin{frame}{Resumen}

El algoritmo consta de 3 pasos importantes:

\begin{enumerate}
\item Ordenar los puntos por su coordenada x (para poder hacer el barrido con una l\'inea vertical). Esto toma $O(n \lg n)$.
\item Insertar y remover cada punto una vez del set ordenado por coordenada $y$, que toma $O(n \lg n)$ (insertar y remover un elemento toma $O(\lg n)$ usando un set, y hay $n$ puntos en total).
\item Comparar cada punto con una cantidad constante de candidatos ($\leq 6$), lo que toma $O(n)$ en total.
\end{enumerate}
   
As\'i, en total el algoritmo es $O(n \lg n)$!
\end{frame}

\begin{frame}{Problema para pensar}
http://www.spoj.com/problems/CLOSEST/
\end{frame}

\subsubsection{Intersecci\'on de segmentos}

\begin{frame}
\begin{block}{Problema}
Dados $n$ segmentos en el plano (horizontales o verticales), queremos hallar todas las intersecciones entre dos segmentos.
\end{block}

\bigskip
Nuevamente, si entra en tiempo la soluci\'on trivial $O(n^2)$, hay que programar esa.
\bigskip

Si no, como es por ejemplo el caso $n = 1000000$, tenemos que pensar otra manera de resolverlo. Utilizaremos sweep line.

\end{frame}

\begin{frame}
\begin{itemize}
\item De nuevo utilizaremos un barrido con una l\'inea vertical, pero aqu\'i la situaci\'on es distinta: no ser\'an puntos los que crucen la l\'inea, sino segmentos.
\bigskip
\item Para representarlos, utilizaremos \textbf{eventos}. Los eventos ser\'an coordenadas $x$ en los que pasa algo relevante:
\bigskip
\begin{itemize}
\item Por cada segmento horizontal, habr\'a un evento de apertura (cuando comienza el segmento) y uno de clausura (cuando pasamos por el final del mismo). 
\bigskip

\item Por cada segmento vertical, habr\'a un evento (la aparici\'on del segmento en una cierta coordenada $x$).
\end{itemize}
\end{itemize}
\end{frame}

\begin{frame}[fragile]{Pseudoc\'odigo del algoritmo}

\begin{lstlisting}
lista = la lista de todos los eventos posibles
ordeno lista por las coordenadas x de sus elementos

S = set de segmentos ordenado por la coordenada y de sus elementos
    (inicialmente vacio)

para cada event en lista
    if ( event = horizontal de apertura )
        inserto el segmento en el set S
    if ( event = horizontal de clausura )
        quito el segmento del set S
    if ( event = vertical (segmento de y1 a y2, con y1 < y2) )
        itero los segmentos de S en el rango (y1,y2)
        imprimo las intersecciones halladas

\end{lstlisting}

\textbf{Detalle:} en un set se puede hallar el rango (y1,y2) en $O(\lg n)$,
utilizando las funciones \texttt{lowerbound} y \texttt{upperbound}.

\end{frame}

\begin{frame}{Problema para pensar}
http://goo.gl/UT0O2U (A Safe Bet - WF 2012)
\end{frame}

\subsubsection{Área de unión de rectángulos}
\begin{frame}{Área de unión de rectángulos}
%%%%%%%%%%% TODO %%%%%%%%%%%%%

\end{frame}


\subsection{Sweep circle}

\begin{frame}{¿Qu\'e es sweep circle?}
\begin{itemize}
\item La idea de sweep circle es exactamente la misma que la ya mencionada sweep line, pero en lugar de mover una recta imaginaria, movemos una circunferencia.

\item La circunferencia puede moverse en linea recta (traslación) o alrededor de un centro fijo (rotación).

\item Como un círculo es una figura acotada, el choque de la circunferencia con los puntos interesantes produce eventos de \textit{entrada} y de \textit{salida} en el círculo.
\end{itemize}
\end{frame}

\begin{frame}{Ejemplo: Ubicación ideal de un círculo en el eje Y}

\begin{block}{Problema}
    Dado un radio $R > 0$ entero, se debe indicar cuál es la máxima cantidad de puntos de la grilla de coordenadas enteras que es posible encerrar con un círculo de radio $R$, cuyo centro se encuentre posicionado sobre la recta $x=0$ (el eje $y$).
\end{block}

\pause
\invisible<1-1>
{
    Observación: Alcanza con considerar las posiciones $0 \leq y \leq 1$
}

\end{frame}

\begin{frame}{Planteo con sweep circle}

\begin{itemize}
    \item Comenzamos con el círculo ubicado en $(0,0)$, y todos los correspondientes puntos de la grilla adentro.
    \item ``Movemos'' el círculo en vertical, hasta llegar a $(0,1)$, procesando los eventos de entrada y salida de puntos.
    \item La máxima cantidad de puntos que tengamos dentro del círculo en cualquier momento, es el resultado.
\end{itemize}

\pause

\invisible<1-1>{
    \begin{itemize}
       \item Notar que hay solamente $O(R)$ eventos de entrada / salida, y además la cantidad de puntos totales dentro del círculo inicial puede computarse en $O(R)$.
       \item Con todo esto y la técnica de barrido, el problema se resuelve en $O(R \lg R)$
    \end{itemize}
}

\end{frame}

\subsection{Sweep ray (o semirrecta)}
\begin{frame}{C\'apsula convexa (Convex Hull)}
\begin{block}{Problema}
Se tiene un mill\'on vacas en un campo. Se quieren encerrar todas las vacas con una cerca, de manera tal que se minimice la cantidad de alambre utilizado (las vacas deben quedar en el borde o dentro de la cerca).
\end{block}

¿Ideas? \bigpause

\invisible<1>{
Para resolver el problema, tendremos que introducir dos conceptos antes.
}
\end{frame}

\begin{frame}{Pol\'igonos convexos}
\begin{block}{Pol\'igono convexo}
Un pol\'igono es convexo si cumple que dados dos puntos cualesquiera en su interior, el segmento que los une est\'a completamente contenido en \'el.
\end{block}

\begin{center}
\includegraphics[scale=0.4]{images/convexo.png}
\end{center}

Equivalentemente, un pol\'igono es convexo si los todos sus \'angulos interiores son menores que $180^\circ$.

\bigpause
\invisible<1>{
Notar que el pol\'igono que ser\'a soluci\'on al problema debe ser convexo.
}
\end{frame}


\begin{frame}
\begin{block}{C\'apsula convexa}
Dados $n$ puntos, la \emph{c\'apsula convexa} es el menor pol\'igono convexo que contiene a todos los puntos 
en su interior. Se puede probar que es \'unica.
\end{block}

\begin{center}
\includegraphics[scale=0.5]{images/convex_hull.png}
\end{center}

\invisible<1>{
\pause 
Entonces, podemos ver que el problema se reduce a hallar la c\'apsula convexa. ¿Pero c\'omo lo hacemos?
}
\end{frame}

\begin{frame}{Algoritmo de Graham Scan}

Se elige el punto de m\'as abajo (con menor $y$), y en caso de haber m\'as de un punto con la misma $y$ se elige el punto de m\'as a la izquierda (o sea, con menor $x$). A este punto lo llamaremos $P$. Elegirlo tiene una complejidad $O(n)$.

\bigskip

\begin{center}
\includegraphics[scale=0.5]{images/convex_hull1.png}
\end{center}

\end{frame}

\begin{frame}
\textbf{Iremos barriendo el plano con una semirrecta que parta desde $P$}. Esta semirrecta no ser\'a vertical ni horizontal, como ven\'iamos viendo, sino que \textbf{girar\'a en sentido antihorario}.

\bigpause
\invisible<1>{
Para poder hacerlo, tenemos que ordenar los puntos por su \'angulo respecto a $P$ y al eje $x$. Si dos puntos tienen el mismo \'angulo, desempatamos por el m\'as cercano primero.

\begin{center}
\includegraphics[scale=0.5]{images/convex_hull2.png}
\end{center}
}
\end{frame}

\begin{frame}
Ahora veamos c\'omo manejamos los eventos (en este caso, ser\'a simplemente la aparici\'on de un punto en el barrido).\bigskip

El pr\'oximo punto siempre est\'a en la Convex Hull. En este caso, $u\times v > 0$.

\begin{center}
\includegraphics[scale=0.5]{images/convex_hull3.png}
\end{center}

\end{frame}

\begin{frame}
Mientras que $u\times v < 0$, sacamos el punto anterior (¡forma un \'angulo mayor que $180^\circ$!).

\begin{center}
\includegraphics[scale=0.5]{images/convex_hull4.png}
\end{center}

\end{frame}

\begin{frame}[fragile]{Pseudoc\'odigo del Graham Scan}

\begin{lstlisting}
vector<Punto> ConvexHull ( vector<Punto>& lista ) {
    if( |lista| < 3 ) devolver lista
    p = el punto de mas abajo y mas a la izquierda
    Ordenar todos los puntos por angulo respecto a p
        (desempatando por distancia a p, los mas cercanos primero)
    pila = stack de puntos vacio
    pila.push(lista[0]) // lista[0] == p
    pila.push(lista[1])
    i = 2
    while (i < N)
        sea a el tope de pila, b el anteultimo de la pila (si existen)
        if (pila.size > 1 && pcruz(lista[i] - a , b - a) <= 0 )
            pila.pop()
        else
            pila.push(lista[i])
            i = i + 1
    devolver pila
}
\end{lstlisting}

\end{frame}

\begin{frame}{Problemas para pensar}

\begin{itemize}
\item goo.gl/rT7Ji
\item goo.gl/llEHC
\end{itemize}

\end{frame}

\section{Planaridad (SECCION DE AGUS SIN REVISAR)}

\subsection{Definiciones}

\begin{frame}{Grafo planar}

\begin{block}{Definición}
    Un grafo se dice \textit{planar} si es posible dibujarlo en el plano, haciendo corresponder a cada vértice un punto, y a cada arista una curva simple continua que una los puntos correspondientes a los extremos de la arista, de manera tal que dos curvas correspondientes a aristas distintas no se intersequen más que en sus extremos.
\end{block}

\begin{block}{Definición}
    Dado un grafo planar $G$, a un dibujo de $G$ en el plano que cumple lo enunciado en la definición anterior se lo denomina un \textit{embedding}, \textit{inmersión} o simplemente \textit{dibujo} de $G$.
\end{block}

Notar que un mismo grafo planar $G$ puede tener infinitos embeddings distintos.

\end{frame}

\begin{frame}{Región}

\begin{block}{Definición}
    Dado un embedding $E$ de un grafo planar $G$, se denomina una \textit{región} de $E$ a una componente conexa del conjunto de puntos del plano que no forman parte del dibujo de $G$ en $E$.
\end{block}

Notar que al igual que muchas otras propiedades de un dibujo de un grafo planar, el conjunto de regiones depende del dibujo, y \textbf{en principio}, distintos dibujos de un mismo grafo planar podrían tener diferente cantidad de regiones.

\end{frame}

\begin{frame}{Región (cont)}

\begin{block}{Definición}
    Dada una región $f$ de un dibujo de un grafo planar $G$, se denomina el \textit{grado} de $f$ y lo notaremos $d(f)$, a la cantidad de aristas presentes en la \textit{frontera} de $f$ en el dibujo. Además, si la región $f$ toca
    a la arista de ambos lados, entonces será contada dos veces para el grado.
\end{block}

Notar que de la definición surge que cada arista ``aporta grado'' a exactamente dos regiones (o bien, a una misma región dos veces), de donde siempre se tiene $\sum_f {d(f)} = 2m$

\end{frame}

\subsection{Fórmula de Euler}

\begin{frame}{Fórmula de Euler}

La principal herramienta para trabajar con grafos planares es el siguiente resultado:

\begin{block}{Teorema}
    Si $G$ es un grafo planar conexo de $n$ vértices y $m$ aristas, y $R$ es la cantidad de regiones de \textbf{cualquier} dibujo de $G$, entonces:    
    $$R + n = m + 2 \mbox{ (fórmula de Euler)}$$
    
    En general, para un grafo planar cualquiera con $c \geq 1$ componentes conexas vale:    
    $$R + n = m + c + 1$$
\end{block}

Observar que esto es válido incluso si el grafo contiene multiejes (más de un eje entre un mismo par de nodos) y bucles (ejes de un nodo a sí mismo).

\end{frame}

\begin{frame}{Raleza de los grafos planares}

Sea $G$ un grafo simple (sin multiejes ni bucles) planar, y $g$ la longitud mínima de un ciclo simple de $g$ (si $G$ no tiene ciclos tendremos directamente $m \leq n-1$).

\begin{block}{Teorema}
    Si $G$ cumple lo anterior, entonces $m \leq \frac{(n-c-1)g}{g-2}$.
\end{block}
\begin{block}{Corolario}
    Si $G$ es grafo simple planar con $n \geq 3$, entonces $m \leq 3n - 6$.
\end{block}

Para demostrar esto, notamos que la frontera de una región debe contener un circuito, así que
$$2m = \sum_f {d(f)} \geq R g = (m-n+c+1) g \Rightarrow m \leq \frac{(n-c-1)g}{g-2}$$

\end{frame}



\begin{frame}{Ejemplos mínimos de grafos no planares}

Como consecuencia de lo anterior, notamos que:

\begin{itemize}
    \item $K_5$ no es planar: tiene $m=10$ y $n=5$, y no cumple $m \leq 3n-6$.
    \item $K_{3,3}$ no es planar: tiene $m=9$ , $n=6$ , $g=4$ y $c=1$, y no cumple $m \leq \frac{(n-c-1)g}{g-2} = \frac{(6-1-1)4}{2} = 8$.
\end{itemize}

Estos son los ejemplos no planares con menor cantidad de nodos y aristas, respectivamente.

\end{frame}

\begin{frame}{Referencias}
   \begin{itemize}
   \item \textit{Introduction to Algorithms, 2nd Edition}. MIT Press. \\ Thomas H. Cormen
   \textbf{Sección 33} (Computational Geometry)
   \item \url{https://www.topcoder.com/tc?module=Static&d1=tutorials&d2=geometry1}
   \item \url{https://www.topcoder.com/tc?module=Static&d1=tutorials&d2=geometry2}
   \item \url{https://www.topcoder.com/tc?module=Static&d1=tutorials&d2=geometry3}
   \item \url{https://www.topcoder.com/tc?module=Static&d1=tutorials&d2=lineSweep}
  \end{itemize}
  
\end{frame}


\end{document}
