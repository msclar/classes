\documentclass[compress]{beamer}

\mode<presentation>
{
  %\usetheme{Warsaw}
  %\usecolortheme{spruce}
  % or ...
	%\useoutertheme{infolines}
  %\setbeamercovered{transparent}
  
  \usetheme{CambridgeUS}
    \setbeamercolor{item projected}{bg=darkred}
    \setbeamertemplate{enumerate items}[default]
    \setbeamertemplate{navigation symbols}{}
    \setbeamercovered{transparent}
    \setbeamercolor{block title}{fg=darkred}
    \setbeamercolor{local structure}{fg=darkred}
  
  % or whatever (possibly just delete it)
}

\usepackage{verbatim} 
\usepackage{listings}
\usepackage{tikz}
\usetikzlibrary{arrows}
\usetikzlibrary{shapes}
\tikzstyle{block}=[draw opacity=0.7,line width=1.4cm]

\newcommand{\bigpause}{\bigskip \pause}

\lstloadlanguages{C++}
\lstnewenvironment{code}
	{%\lstset{	numbers=none, frame=lines, basicstyle=\small\ttfamily, }%
	 \csname lst@SetFirstLabel\endcsname}
	{\csname lst@SaveFirstLabel\endcsname}
\lstset{% general command to set parameter(s)
	language=C++, basicstyle=\footnotesize\sffamily, keywordstyle=\slshape,
	emph=[1]{tipo,usa}, emphstyle={[1]\sffamily\bfseries},
	basewidth={0.47em,0.40em},
	columns=fixed, fontadjust, resetmargins, xrightmargin=5pt, xleftmargin=15pt,
	flexiblecolumns=false, tabsize=2, breaklines,	breakatwhitespace=false, extendedchars=true,
	numbers=left, numberstyle=\tiny, stepnumber=1, numbersep=9pt,
	frame=l, framesep=3pt,
}

\usepackage[spanish]{babel}
% or whatever

\usepackage[utf8]{inputenc}
% or whatever

\usepackage{times}
\usepackage[T1]{fontenc}
% Or whatever. Note that the encoding and the font should match. If T1
% does not look nice, try deleting the line with the fontenc.


\title[Geometr\'ia Computacional] % (optional, use only with long paper titles)
{Geometr\'ia Computacional}

\author[Melanie Sclar] % (optional, use only with lots of authors)
{~Melanie Sclar}
% - Give the names in the same order as the appear in the paper.
% - Use the \inst{?} command only if the authors have different
%   affiliation.
\institute[UBA] % (optional, but mostly needed)
{
  %\inst{1}%
  Facultad de Ciencias Exactas y Naturales\\
  Universidad de Buenos Aires
}
\date[TC 2014] % (optional, should be abbreviation of conference name)
{Training Camp 2014}

% Ac¿ se puede insertar el logo de la UBA
% \pgfdeclareimage[height=0.5cm]{university-logo}{university-logo-filename}
% \logo{\pgfuseimage{university-logo}}



% Delete this, if you do not want the table of contents to pop up at
% the beginning of each subsection:
\AtBeginSubsection[]
{
  \begin{frame}<beamer>{Contenidos}
    \tableofcontents[currentsection,currentsubsection]
  \end{frame}
}

\newcommand{\be}{\begin{equation*}}
\newcommand{\ee}{\end{equation*}}
\newcommand{\state}[1]{\left|\,#1\,\right\rangle}
\newcommand{\costate}[1]{\left\langle\,#1\,\right|}
\newcommand{\trace}{\text{Tr}}
\newcommand{\su}{\uparrow}
\newcommand{\sd}{\downarrow}
\newcommand{\im}{\text{Im}}
\newcommand{\re}{\text{Re}}

% If you wish to uncover everything in a step-wise fashion, uncomment
% the following command:

%\beamerdefaultoverlayspecification{<+->}


\begin{document}
\pgfdeclarelayer{background}
\pgfsetlayers{background,main}
\begin{frame}
  \titlepage
\end{frame}

\begin{frame}{Contenidos}
  \tableofcontents
  % You might wish to add the option [pausesections]
\end{frame}

\begin{frame}{Objetivo del trabajo}

El objetivo de este trabajo es entender que las manipulaciones que los
votantes hacen en una elección puede afectar el resultado de la misma
más frecuentemente de lo que pensamos.\\
\bigskip

Formalizaremos las nociones de ``elección'' y ``manipulación'' para luego
enunciar el teorema a probar, y finalmente probarlo.

\end{frame}

\begin{frame}{Definiciones}
Supongamos que hay $m$ candidatos en la elección, y sea $[m]$ el conjunto
que contiene a estas $m$ alternativas. Sea $L$ es el conjunto de órdenes totales sobre $[m]$.\\
\bigskip

En este modelo, cada votante tendrá un vector de $m$ elementos representando
el orden total de sus preferencias entre los $m$ candidatos.\\
\bigskip

Una \textit{social choice function} (SCF) es una función $f : L^n \rightarrow [m]$,
que toma los $n$ vectores de preferencias de los votantes y elige uno de
los candidatos.

\end{frame}

\begin{frame}

\begin{block}{Definición de manipulación}
Una \textit{manipulación (útil)} de una SCF $f$ realizada por un
votante $i$ sobre el perfil $(x_1, \ldots, x_n)$ es una preferencia $x_i'$ 
tal que el resultado $f(x_i', x_{-i})$ es preferido por el votante $i$
al resultado original $f(x_i,x_{-i})$.
\end{block}
\bigskip

Intuitivamente, una manipulación consiste en que un votante miente al
mostrar su vector de preferencias y con esta mentira obtiene un rédito:
el resultado final de la elección le gustará más que si no hubiera mentido.

\end{frame}

\begin{frame}

\begin{block}{Definición de poder de manipulación}
El \textit{poder de manipulación} de un votante $i$ en una SCF $f$ 
es la probabilidad de que $x_i'$ sea una manipulación útil de $f$
sobre el perfil $(x_1, \ldots, x_n)$, donde $x_1, \ldots, x_n$ y $x_i'$
son elegidos al azar uniformemente entre todos los posibles órdenes
totales de $[m]$.\\
Lo notamos como $M_i(f)$.
\end{block}
\bigskip

Esta no es la definición más precisa, pues asume una distribución uniforme
sobre las preferencias. Si bien esto no es realista, el objetivo del
trabajo es lograr una cota inferior para el poder de manipulación (¡y ver
que es no despreciable!).
\end{frame}

\begin{frame}{Noción de distancia entre funciones}

Dadas dos funciones $f$ y $g$ que van de un espacio de probabilidad $X$
a un conjunto $Y$, se define la distancia entre ambas como

$$ \Delta(f, g) = Pr_{x \in X} [f(x) \neq g(x)] $$

Si quiero definir la distancia de una función $f$ a un conjunto $G$ de
funciones (como podrían serlo las dictaduras), definimos

$$ \Delta(f, G) = min_{g \in G} \Delta(f, g) $$

\end{frame}

\begin{frame}{Teorema principal del trabajo}

\begin{block}{Teorema principal}
Existe una constante $C > 0$ tal que para todo $\epsilon > 0$, si $f$ es
una SCF neutral tal que $\Delta(f, Dictaduras) > \epsilon$ y que 
elige entre 3 candidatos ($m = 3$), 
entonces $\sum_{i=1}^n M_i(f) \geq C \epsilon^2$
\end{block}
\bigskip

$Nota \ 1:$ se dice que $f$ es neutral si los nombres de los candidatos no
influyen en la decisión final. Es decir, si $\sigma$ es una permutación
vale que $f(\sigma(x_1), \ldots \sigma(x_n)) = \sigma(f(x_1,\ldots,x_n))$.\\
\bigskip

$Nota \ 2:$ se dice que $g$ es una dictadura si es una SCF donde 
se elige siempre el primer elemento del vector de preferencias de 
algún votante.

\end{frame}

\begin{frame}{Implicancias del teorema}
\begin{itemize}
\item Con la prueba de este teorema se puede ver que $max_i \ M_i(f) \geq 
\Omega(1/ \sqrt n)$.

\item No se puede acotar por una cota que no dependa de
$n$ pues para el método de voto plural vale que 
$M_i(f) = \Theta(1/\sqrt n)$

\item Los autores dejan abierto el problema para $m > 3$
\end{itemize}
\end{frame}

\begin{frame}
\end{frame}


\begin{frame}{¡Probémoslo!}

Para probar el teorema daremos un approach "top down". Hay 2 lemas que
juntos nos permiten implicar lo buscado. Primero los enumeraremos y 
entenderemos por qué resuelven el problema, y luego los probaremos uno
por uno.

\end{frame}

%\begin{frame} {Lema 1}
%Para todo $m$ y $\epsilon > 0$ existe un $\delta > 0$ tal que si 
%$F = f^{\otimes \ \binom{m}{2}}$ es una IAA GSWF neutral sobre $m$
%alternativas con $f : \{0,1\}^n \rightarrow \{0,1\}$, $\Delta(f, DICT) > \epsilon$
%\end{frame}

\begin{frame}
\begin{block}{Lema 2}
Para todo $m$ existe $\delta > 0$ tal que para todo $\epsilon > 0$ 
vale que si $f$ es una SCF neutral entre $m$ candidatos tal que 
$\Delta(f, Dictaduras \ \cup \ AntiDictaduras) > \epsilon$
entonces para todo $(a, b)$ se tiene que $M^{a,b}(f) \geq \delta$.\\
\bigskip

Además, para $m = 3$ vale que $\delta = C\epsilon^2$ donde $C>0$ es una
constante.
\end{block}
\bigskip

\begin{block}{Lema 3}
Para toda SCF $f$ con $m = 3$ y todo $a, b \in A$, 
$M^{a,b}(f) \leq 6 \sum_i M_i(f)$
\end{block}
\end{frame}

\begin{frame}
Sea $f$ una SCF neutral con $m = 3$ tal que $\Delta(f, Dictaduras) > \epsilon$.

\begin{itemize}
\item Si $\Delta(f, AntiDictaduras) > \epsilon$: \\
$$ C\epsilon^2 \leq M^{a,b}(f) \leq 6 \sum_i M_i(f)$$ 
\item Si $\Delta(f, AntiDictaduras) \leq \epsilon$
\end{itemize}

%%%%%%%%%%%%%%%%%%% NO ME SALE %%%%%%%%%%%%%%%%%%
%% Es un argumento de la pinta de principio de palomar, 

QED.
\end{frame}

\begin{frame}
Ahora resta probar los dos lemas: comencemos por el lema 2. Para ello,
necesitaremos un teorema probado en otro trabajo, pero antes de presentarlo
explicamos las definiciones necesarias para entender su enunciado.

%La consecuencia de este teorema puede ser reescrita como 
%$\Delta(F, CW) \geq \delta$, pues $\Delta(f, CW)$ representa la 
%probabilidad de que $f$ no coincida con un elemento de $CW$.

%%%%%%%%%%%%%%%%%%%%%%% DEFINIR %%%%%%%%%%%%%%%%
% GSWF
% IIA
% CW

\end{frame}

\begin{frame}

\begin{block}{Lema 1 - consecuencia directa de $[Ka02]$}
Para todo $m$ y $\epsilon > 0$ existe un $\delta > 0$ tal que si 
$F = f^{\otimes \ \binom{m}{2}}$ es una IIA GSWF neutral sobre $m$
alternativas con $f : \{0,1\}^n \rightarrow \{0,1\}$ entonces vale que
$$\Delta(f, Dictaduras \cup AntiDictaduras) > \epsilon
\Rightarrow \Delta(F, CW) \geq \delta$$
\bigskip
Para $m=3,4,5$ vale que $\delta = C \epsilon$ con $C > 0$ una constante.
\end{block}

\begin{block}{Lema 1' - contrapositiva de Lema 1 para $m = 3$}
Para todo $\epsilon > 0$ existe un $\delta > 0$ tal que si 
$F = f^{\otimes \ \binom{m}{2}}$ es una IIA GSWF neutral con 
$f : \{0,1\}^n \rightarrow \{0,1\}$ entonces vale que

$$ \Delta(F, CW) > C \epsilon \Rightarrow \Delta(f, Dictaduras \cup AntiDictaduras) \leq \epsilon$$
\end{block}
\end{frame}

\begin{frame}
En vez de probar $Lema \ 2$, vamos a probar su contrapositiva también.

\begin{block}{Lema 2' - contrapositiva de Lema 2}
Existe $\delta > 0$ tal que para todo $\epsilon > 0$ si $f$ es una SCF
neutral entonces para todo $a, b$ vale que

$$ M^{a,b}(f) \leq C \epsilon^2 \Rightarrow \Delta(f, Dictaduras \cup AntiDictaduras) \leq \epsilon$$
\end{block}
\end{frame}

\begin{frame}{Hoja de ruta para el lema 2}
\begin{itemize}
\item Dada una SCF $f$ con $M^{a,b}(f) \leq C \epsilon^2$ para todo $a,b$
\end{itemize}
\end{frame}

\begin{frame}
\end{frame}

\begin{frame}{Prueba del lema 3 - siguen las definiciones}
$$A(z^{a,b}) := \{x | x^{a,b} = z^{a,b}, f(x) = a\}$$
$$B(z^{a,b}) := \{x | x^{a,b} = z^{a,b}, f(x) = b\}$$

\end{frame}

\begin{frame}
\end{frame}

\end{document}
