\documentclass[compress]{beamer}

\mode<presentation>
{
  %\usetheme{Warsaw}
  %\usecolortheme{spruce}
  % or ...
	%\useoutertheme{infolines}
  %\setbeamercovered{transparent}
  
  \usetheme{CambridgeUS}
    \setbeamercolor{item projected}{bg=darkred}
    \setbeamertemplate{enumerate items}[default]
    \setbeamertemplate{navigation symbols}{}
    \setbeamercovered{transparent}
    \setbeamercolor{block title}{fg=darkred}
    \setbeamercolor{local structure}{fg=darkred}
  
  % or whatever (possibly just delete it)
}

\usepackage{verbatim} 
\usepackage{listings}
\usepackage{tikz}
\usetikzlibrary{arrows}
\usetikzlibrary{shapes}
\usepackage{parskip}
\tikzstyle{block}=[draw opacity=0.7,line width=1.4cm]
\setlength{\parindent}{0cm}
\newcommand{\bigpause}{\bigskip \pause}

\lstloadlanguages{C++}
\lstnewenvironment{code}
	{%\lstset{	numbers=none, frame=lines, basicstyle=\small\ttfamily, }%
	 \csname lst@SetFirstLabel\endcsname}
	{\csname lst@SaveFirstLabel\endcsname}
\lstset{% general command to set parameter(s)
	language=C++, basicstyle=\footnotesize\sffamily, keywordstyle=\slshape,
	emph=[1]{tipo,usa}, emphstyle={[1]\sffamily\bfseries},
	basewidth={0.47em,0.40em},
	columns=fixed, fontadjust, resetmargins, xrightmargin=5pt, xleftmargin=15pt,
	flexiblecolumns=false, tabsize=2, breaklines,	breakatwhitespace=false, extendedchars=true,
	numbers=left, numberstyle=\tiny, stepnumber=1, numbersep=9pt,
	frame=l, framesep=3pt,
}

\usepackage[spanish]{babel}
% or whatever

\usepackage[utf8]{inputenc}
% or whatever

\usepackage{times}
\usepackage[T1]{fontenc}
% Or whatever. Note that the encoding and the font should match. If T1
% does not look nice, try deleting the line with the fontenc.


\title[Programación dinámica] % (optional, use only with long paper titles)
{Programación dinámica}

\author[Melanie Sclar] % (optional, use only with lots of authors)
{~Melanie Sclar}
% - Give the names in the same order as the appear in the paper.
% - Use the \inst{?} command only if the authors have different
%   affiliation.
\institute[UBA] % (optional, but mostly needed)
{
  %\inst{1}%
  Facultad de Ciencias Exactas y Naturales\\
  Universidad de Buenos Aires
}
\date[AED III] % (optional, should be abbreviation of conference name)
{AED III}

% Ac¿ se puede insertar el logo de la UBA
% \pgfdeclareimage[height=0.5cm]{university-logo}{university-logo-filename}
% \logo{\pgfuseimage{university-logo}}



% Delete this, if you do not want the table of contents to pop up at
% the beginning of each subsection:
\AtBeginSubsection[]
{
  \begin{frame}<beamer>{Contenidos}
    \tableofcontents[currentsection,currentsubsection]
  \end{frame}
}

\newcommand{\be}{\begin{equation*}}
\newcommand{\ee}{\end{equation*}}
\newcommand{\state}[1]{\left|\,#1\,\right\rangle}
\newcommand{\costate}[1]{\left\langle\,#1\,\right|}
\newcommand{\trace}{\text{Tr}}
\newcommand{\su}{\uparrow}
\newcommand{\sd}{\downarrow}
\newcommand{\im}{\text{Im}}
\newcommand{\re}{\text{Re}}


\newcommand{\Asig}{\ensuremath{\leftarrow}}
\newcommand{\AndY}{\ensuremath{\wedge}}
\newcommand{\Or}{\ensuremath{\vee}}
\newcommand{\Not}{\ensuremath{\neg}}
\newcommand{\NotEq}{\ensuremath{\neq}}
\newcommand{\MayorIg}{\ensuremath{\geq}}
\newcommand{\tabu}{\hspace*{0.7cm}}
\newcommand{\ctabu}{\hspace*{0.8cm}}
\newcommand{\htabu}{\hspace*{0.25cm}}
\newcommand{\moduloNombre}[1]{\textbf{#1}}

% If you wish to uncover everything in a step-wise fashion, uncomment
% the following command:

%\beamerdefaultoverlayspecification{<+->}

\begin{document}

\pgfdeclarelayer{background}
\pgfsetlayers{background,main}
\begin{frame}
  \titlepage
\end{frame}


\begin{frame}{Soluciones recursivas a problemas}

  \begin{itemize}
   \item Muchos algoritmos de utilidad son recursivos: para resolver un problema, se utilizan las soluciones a subproblemas fuertemente relacionados.
   \item En estos algoritmos, se divide el problema en varios subproblemas 
   que luego se resuelven y se combinan las soluciones obtenidas para resolver el original.
   \item Un ejemplo de técnica recursiva de diseño de algoritmos es la técnica de Divide and Conquer, vista en algoritmos 2.
  \end{itemize}
  
\end{frame}

\begin{frame}{¿En qué consiste la programación dinámica?}
\small
  \begin{itemize}
   \item La programación dinámica es una técnica de solución de problemas recursiva.
   \item Al igual que Divide and Conquer, la técnica propone descomponer el problema a resolver en \textbf{subproblemas más pequeños de la
   misma especie}, para resolverlos recursivamente y combinar esas soluciones en una solución al problema original.
   \item La diferencia esencial que lo contrasta con Divide and Conquer, es que mientras que en esta técnica los subproblemas que se resuelven
    son independientes entre sí y se resuelven individualmente, la programación dinámica es aplicable cuando los subproblemas \textbf{no son independientes}.
   \item En estos casos, un algoritmo de Divide and Conquer realizaría el mismo trabajo múltiples veces, ya que la solución a un mismo subproblema puede
   ser \textbf{recalculada} muchas veces si se la reutiliza como parte de varios subproblemas más grandes.
  \end{itemize}
    
\end{frame}

\begin{frame}{¿En qué consiste la programación dinámica? (2)}
  \begin{itemize}
   \item La solución que propone la técnica de programación dinámica es \textbf{almacenar} las soluciones a subproblemas ya calculadas, de
   manera de calcularlas una sola vez, y luego leer el valor ya calculado cada vez que se lo vuelve a necesitar.
   \item Uno de los usos más importantes de esta técnica es en problemas de \textbf{optimización}: En estos problemas interesa
   encontrar la solución que maximiza un cierto puntaje u objetivo, en un espacio de soluciones posibles.
   \item Un indicador central de la aplicabilidad de la técnica lo constituye el \textbf{principio del óptimo}. Este principio afirma
    que \textbf{las partes de una solución óptima} a un problema, deben ser \textbf{soluciones óptimas de los correspondientes subproblemas}, y es lo que
    permite obtener una solución óptima al problema original a partir de soluciones óptimas de los subproblemas.
  \end{itemize}
    
\end{frame}

\begin{frame}{El esquema general}
    Los algoritmos de programación dinámica se pueden organizar típicamente en 4 pasos que responden al siguiente esquema general:
   \begin{enumerate}
   \item Caracterizar la estructura de una solución óptima.
   \item Definir recursivamente el valor de una solución óptima.
   \item Computar el \textbf{valor} de una solución óptima. Se calcula de manera \textit{bottom-up}.
   \item Construir una solución óptima a partir de la información obtenida en el paso 3
  \end{enumerate}
    El paso 4 es optativo, ya que si solo nos interesa el valor o puntaje de una solución óptima pero no la solución en sí, este paso
    de reconstrucción no es necesario.
\end{frame}

\begin{frame}{Problema del recorrido óptimo en una matriz}

\small
	Sea $M \in \mathbb{N}^{m \times n}$ una matriz de números naturales. Se desea obtener un camino que empiece en la casilla superior
    izquierda $(1,1)$, termine en la casilla inferior derecha $(m,n)$ y tal que minimice la suma de los valores de las casillas por
    las que pasa. En cada casilla $(i,j)$ hay dos movimientos posibles: ir hacia abajo (a la casilla $(i+1, j)$), o ir hacia la
    derecha (a la casilla $(i, j+1)$).
   \begin{itemize}
   \item[a] Diseñar un algoritmo eficiente basado en programación dinámica que resuelva este problema.
   \item[b] Determinar la complejidad del algoritmo propuesto (temporal y espacial).
   \item[c] Exhibir el comportamiento del algoritmo sobre la matriz que aparece a continuación.
  \end{itemize}
                \[ \left [  \begin{array}{cccc}
2 & 8 & 3 & 4 \\
5 & 3 & 4 & 5 \\
1 & 2 & 2 & 1 \\
3 & 4 & 6 & 5 \\
 \end{array} \right ] \]

\end{frame}

\begin{frame}{Fórmula recursiva}
    \small
    La matriz de resultados parciales almacena en $best(i,j)$ la mínima longitud de un camino que empiece en $(1,1)$ y llegue
    a $(i,j)$, haciendo solo movimientos hacia abajo y hacia la derecha.
   \begin{itemize}
   \item $best(i,j) = M_{i,j} + min(best(i-1,j), best(i,j-1))$ \\ \ \ \ \ \ \ \ \ \ \ \ \ \ \ \ \ \ \ \ \ \ \ \ \ \ \ \ \ \ \ \ \ \ \ \ \ \ \ \ \ \ \ \ \ \ \ \ \ \ \ \ \ \ \ para $1 < i \leq m$ y $1 < j \leq n$
   \item $best(i,1) = M_{i,1} + best(i-1,1)$ para $1 < i \leq m$
   \item $best(1,j) = M_{1,j} + best(1,j-1)$ para $1 < j \leq n$
   \item $best(1,1) = M_{1,1}$
  \end{itemize}

\end{frame}
\begin{frame}

    La longitud del mínimo camino entre esquinas, que constituye la solución al problema, viene dada por $best(m,n)$. Con esto ya podríamos implementar una solución top-down recursiva:
    \begin{itemize}
       \item Si para calcular un $best(i,j)$ necesitamos un resultado ya calculado, lo usamos directamente.
       \item Sino, lo calculamos recursivamente, almacenamos su valor en la tabla de resultados y luego lo utilizamos.
    \end{itemize}
     
    
\end{frame}


\begin{frame}[fragile]{Algoritmo top-down}
\footnotesize
\begin{lstlisting}
best(Matriz, i, j):
    if (calculado[i][j] != -1)
        return calculado[i][j]
        
    if (i = 1 and j = 1)
        calculado[i][j] <- Matriz[1][1]
    else if (i = 1)
        calculado[i][j] <- best(i, j-1) + Matriz[i][j]
    else if (j = 1)
        calculado[i][j] <- best(i-1, j) + Matriz[i][j]
    else
        calculado[i][j] <- min(best(i-1, j), best(i, j-1)) + Matriz[i][j]
		
    return calculado[i][j]

\end{lstlisting}

La complejidad del algoritmo resultante es $O(nm)$, tanto espacial como temporal. Puede servir especialmente en problemas donde no se necesitar\'a calcular buena parte de todos los estados para obtener el resultado buscado.
\end{frame}


\begin{frame}[fragile]{Algoritmo bottom-up}
\begin{lstlisting}
longitudCaminoMinimo(Matriz, m, n):
    best[1,1] = Matriz[1,1]
    for i = 2 to m do
        best[i,1] = Matriz[i,1] + best[i-1,1]
    for j = 2 to n do
        best[1,j] = Matriz[1,j] + best[1,j-1]
    for i = 2 to m do
        for j = 2 to n do
            best[i,j] = Matriz[i,j] + min(best[i-1,j], best[i,j-1])
    return best[m,n]

\end{lstlisting}

La complejidad del algoritmo resultante es $O(nm)$, tanto espacial como temporal. Se puede bajar la complejidad espacial a $O(min(n,m))$ si no interesa reconstruir el camino sino solo su longitud.
\end{frame}

\begin{frame}{Cálculo en el ejemplo}

Matriz de entrada:
                \[ \left [  \begin{array}{cccc}
\textbf{2} & 8 & 3 & 4 \\
\textbf{5} & 3 & 4 & 5 \\
\textbf{1} & \textbf{2} & \textbf{2} & \textbf{1} \\
3 & 4 & 6 & \textbf{5} \\
 \end{array} \right ] \]

Matriz de best:
                \[ \left [  \begin{array}{cccc}
\textbf{2} & 10 & 13 & 17 \\
\textbf{7} & 10 & 14 & 19 \\
\textbf{8} & \textbf{10} & \textbf{12} & \textbf{13} \\
11 & 14 & 18 & \textbf{18} \\
 \end{array} \right ] \]

En ambas matrices, se indica un camino óptimo en negrita.

\end{frame}

\begin{frame}{Para implementar y seguir pensando}
\begin{itemize}
\item {\small \url{https://projecteuler.net/problem=81}} para implementar el problema que acabamos de ver
\item {\small \url{https://projecteuler.net/problem=67}}, es muy parecido al que ya vimos
\item {\small \url{http://www.oma.org.ar/enunciados/omn20reg.htm}}, problema 3 nivel 1
\end{itemize}
\bigskip
De los primeros dos problemas adjuntamos un c\'odigo que los implementa en sus versiones bottom-up y top-down (pero no los miren antes de intentarlo ustedes!)
\end{frame}

\end{document}