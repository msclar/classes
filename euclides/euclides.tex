\documentclass[compress]{beamer}

\mode<presentation>
{
  %\usetheme{Warsaw}
  %\usecolortheme{spruce}
  % or ...
	%\useoutertheme{infolines}
  %\setbeamercovered{transparent}
  
  \usetheme{CambridgeUS}
    \setbeamercolor{item projected}{bg=darkred}
    \setbeamertemplate{enumerate items}[default]
    \setbeamertemplate{navigation symbols}{}
    \setbeamercovered{transparent}
    \setbeamercolor{block title}{fg=darkred}
    \setbeamercolor{local structure}{fg=darkred}
  
  % or whatever (possibly just delete it)
}

\usepackage{verbatim} 
\usepackage{listings}
\usepackage{tikz}
\usetikzlibrary{arrows}
\usetikzlibrary{shapes}
\usepackage{parskip}
\tikzstyle{block}=[draw opacity=0.7,line width=1.4cm]
\setlength{\parindent}{0cm}
\newcommand{\bigpause}{\bigskip \pause}

\lstloadlanguages{C++}
\lstnewenvironment{code}
	{%\lstset{	numbers=none, frame=lines, basicstyle=\small\ttfamily, }%
	 \csname lst@SetFirstLabel\endcsname}
	{\csname lst@SaveFirstLabel\endcsname}
\lstset{% general command to set parameter(s)
	language=C++, basicstyle=\footnotesize\sffamily, keywordstyle=\slshape,
	emph=[1]{tipo,usa}, emphstyle={[1]\sffamily\bfseries},
	basewidth={0.47em,0.40em},
	columns=fixed, fontadjust, resetmargins, xrightmargin=5pt, xleftmargin=15pt,
	flexiblecolumns=false, tabsize=2, breaklines,	breakatwhitespace=false, extendedchars=true,
	numbers=left, numberstyle=\tiny, stepnumber=1, numbersep=9pt,
	frame=l, framesep=3pt,
}

\usepackage[spanish]{babel}
% or whatever

\usepackage[utf8]{inputenc}
% or whatever

\usepackage{times}
\usepackage[T1]{fontenc}
% Or whatever. Note that the encoding and the font should match. If T1
% does not look nice, try deleting the line with the fontenc.


\title[Algoritmo de Euclides] % (optional, use only with long paper titles)
{Algoritmo de Euclides}

\author[Melanie Sclar] % (optional, use only with lots of authors)
{~Melanie Sclar}
% - Give the names in the same order as the appear in the paper.
% - Use the \inst{?} command only if the authors have different
%   affiliation.
\institute[UBA] % (optional, but mostly needed)
{
  %\inst{1}%
  Facultad de Ciencias Exactas y Naturales\\
  Universidad de Buenos Aires
}
\date[AED III] % (optional, should be abbreviation of conference name)
{AED III}

% Ac¿ se puede insertar el logo de la UBA
% \pgfdeclareimage[height=0.5cm]{university-logo}{university-logo-filename}
% \logo{\pgfuseimage{university-logo}}



% Delete this, if you do not want the table of contents to pop up at
% the beginning of each subsection:
\AtBeginSubsection[]
{
  \begin{frame}<beamer>{Contenidos}
    \tableofcontents[currentsection,currentsubsection]
  \end{frame}
}

\newcommand{\be}{\begin{equation*}}
\newcommand{\ee}{\end{equation*}}
\newcommand{\state}[1]{\left|\,#1\,\right\rangle}
\newcommand{\costate}[1]{\left\langle\,#1\,\right|}
\newcommand{\trace}{\text{Tr}}
\newcommand{\su}{\uparrow}
\newcommand{\sd}{\downarrow}
\newcommand{\im}{\text{Im}}
\newcommand{\re}{\text{Re}}


\newcommand{\Asig}{\ensuremath{\leftarrow}}
\newcommand{\AndY}{\ensuremath{\wedge}}
\newcommand{\Or}{\ensuremath{\vee}}
\newcommand{\Not}{\ensuremath{\neg}}
\newcommand{\NotEq}{\ensuremath{\neq}}
\newcommand{\MayorIg}{\ensuremath{\geq}}
\newcommand{\tabu}{\hspace*{0.7cm}}
\newcommand{\ctabu}{\hspace*{0.8cm}}
\newcommand{\htabu}{\hspace*{0.25cm}}
\newcommand{\moduloNombre}[1]{\textbf{#1}}

% If you wish to uncover everything in a step-wise fashion, uncomment
% the following command:

%\beamerdefaultoverlayspecification{<+->}

\begin{document}

\begin{frame}
  \titlepage
\end{frame}

\begin{frame}{Ejercicio 2.8 de la pr\'actica}
\begin{block}{Ejercicio 2.8}
\begin{enumerate}[a.]
\item Escribir el algoritmo de Euclides para calcular el m\'aximo com\'un divisor entre 2 n\'umeros $b$ y $c$ en forma
recursiva y no recursiva. Mostrar que su complejidad es $O(min\{b,c\})$. Puede construir un ejemplo (un
peor caso) donde esta complejidad efectivamente se alcance? \textquestiondown Puede hacerlo para b y c tan grandes
como se desee?
\end{enumerate}
\end{block}  
\end{frame}

\begin{frame}{M\'aximo Com\'un Divisor}
\begin{block}{Definici\'on del m\'aximo com\'un divisor o $mcd$}
El m\'aximo com\'un divisor entre $a$ y $b$ o $mcd(a,b)$ ($a, b \in \mathbb{N}_0$) es el mayor n\'umero natural que los divide a ambos sin dejar resto. 
\end{block}  

Por ejemplo, $mcd(2,5) = 1$, $mcd(10,30) = 10$, $mcd(55,77) = 11$, $mcd(15,1) = 1$, $mcd(17,0) = 17$. \\
\bigskip

Notar que si $mcd(a,b) = k$ entonces $mcd(\frac{a}{k}, \frac{b}{k}) = 1$ pues quitamos todos los factores primos que eran comunes a ambos n\'umeros.

\end{frame}

\begin{frame}{Algoritmos recursivos}
	\begin{block}{Definici\'on}
	Un algoritmo se dice recursivo si calcula instancias de un problema en funci\'on de otras instancias del mismo problema hasta llegar a un caso base, que suele ser una instancia pequeña del problema, cuya respuesta generalmente est\'a dada en el algoritmo y no es necesario calcularla.
	\end{block}
\bigskip
  Para calcular el $mcd$ de forma recursiva, necesitamos una propiedad del problema que cumpla esto: es decir, que escriba $mcd(a,b)$ en funci\'on de instancias menores que ella hasta llegar a un caso base.
\end{frame}

\begin{frame}{Algoritmos recursivos}
	\begin{block}{Definici\'on}
	Un algoritmo se dice recursivo si calcula instancias de un problema en funci\'on de otras instancias del mismo problema hasta llegar a un caso base, que suele ser una instancia pequeña del problema, cuya respuesta generalmente est\'a dada en el algoritmo y no es necesario calcularla.
	\end{block}
\bigskip
  Para calcular el $mcd$ de forma recursiva, necesitamos una propiedad del problema que cumpla esto: es decir, que escriba $mcd(a,b)$ en funci\'on de instancias menores que ella hasta llegar a un caso base.
\end{frame}

\begin{frame}{Algoritmos recursivos}
	\begin{block}{Lema}
	Sea $a \leq b$ ($a, b \in \mathbb{N}_0$), entonces $mcd(a, b) = k$, $mcd(a, b-a) = k$.
	\end{block}
\bigskip

\pause
\invisible<1>{
{\small 
\textit{\textbf{Demostraci\'on:}} $mcd(a, b) = k$, luego $a = ka'$, $b = kb'$ con $mcd(a', b') = 1$. \\ \bigskip

Veamos que $mcd(a, b-a) = k$: \\
$mcd(a, b-a) = mcd(ka', kb'-ka') = mcd(ka', k(b'-a'))= k \cdot mcd(a', b'-a')$ \\ \bigskip
$\Rightarrow mcd(a, b-a) \geq k \cdot 1 = k$. Entonces seguro que $mcd(a, b-a) \geq k$, pero \textquestiondown puede suceder que $mcd(a, b-a) > k$? \\ \bigskip

Supongamos que $mcd(a, b-a) = q > k$. Entonces tenemos que:

\[ 
\left \{
  \begin{tabular}{c}
  $a \equiv 0 \pmod{q}$\\
  $b-a \equiv 0 \pmod{q}$ 
  \end{tabular}
\right.
\]
\bigskip
Luego concluimos que $a \equiv b \equiv 0 \pmod{q}$.
}
}
\end{frame}

\begin{frame}
Como $a \equiv b \equiv 0 \pmod{q}$, deducimos que $a$ y $b$ son los dos m\'ultiplos de $q$ (y recordemos que por hip\'otesis $q > k$). 
Pero luego $mcd(a,b) \geq q > k$. \\ \bigskip

\textexclamdown Absurdo pues $mcd(a,b) = k$ por hip\'otesis del ejercicio! Provino de suponer que $mcd(a, b-a) > k$. Luego, 
como $mcd(a, b-a) \leq k$ y $mcd(a, b-a) \geq k$ deducimos que $mcd(a, b-a) = k$. $\blacksquare$

\end{frame}

\begin{frame}

\begin{block}{Lema 1}
	Sea $a \leq b$ ($a, b \in \mathbb{N}_0$), entonces $mcd(a, b) = k$, $mcd(a, b-a) = k$.
\end{block}

Este lema es v\'alido como funci\'on recursiva pero puede ser mejorado. Notemos que si $b-a \geq a$, entonces al aplicar de nuevo el Lema obtendremos
$mcd(a,b-2a) = k$, y as\'i sucesivamente hasta que $b-xa < a$. Es decir que podemos mejorar el lema 1 aplic\'andolo varias veces para obtener el lema 2.\\ \bigskip

\pause
\invisible<1>{
\begin{block}{Lema 2}
{\small
	Sea $0 < a \leq b$ ($a, b \in \mathbb{N}_0$), entonces $mcd(a, b) = k$, $mcd(a, b \pmod{a}) = k$.
}
\end{block}
}
\end{frame}

\begin{frame}
\begin{block}{Lema 2}
	Sea $0 < a \leq b$ ($a, b \in \mathbb{N}_0$), entonces $mcd(a, b) = k$, $mcd(a, b \mod{a}) = k$.
\end{block}
\pause
\invisible<1>{
{\small 
\textit{\textbf{Demostraci\'on:}} \\ Como $a \leq b$, $b$ se puede escribir como $b = ra+s$ con $r, s \in \mathbb{N}_0$ y $0 \leq s < a$. \\ \bigskip
Aplicando sucesivamente el Lema 1, por inducci\'on podemos ver que $mcd(a, b-ra) = k$, pues $b - (r-1)a \geq a$ y por ende vale aplicar el lema por $r$-\'esima vez.
Ahora bien, $b-ra = s$, y $s$ es justamente el resto de $b$ m\'odulo $a$. Es decir, $s = b \mod{a}$.\\
En conclusi\'on, queda que $mcd(a,s) = k$, o en otras palabras \\ $mcd(a, b \mod{a}) = k$. $\blacksquare$
}
}
\end{frame}

\begin{frame}
Entonces ya tenemos una propiedad recursiva que nos permitir\'a hallar $mcd(a,b)$. Pero nos falta saber cu\'ando parar, es decir, los \textbf{casos base}. \\ \bigskip
\pause
\invisible<1>{
$mcd(a,0) = a$ \\
$mcd(0,a) = a$ \\ \bigskip

Notemos adem\'as que si no fueran casos base incurrir\'iamos en un error, pues $a\mod{0}$ no se puede efectuar.
}
\end{frame}

\begin{frame}[fragile]{Pseudoc\'odigo de soluci\'on recursiva}
\begin{verbatim}
function mcd (natural a, natural b) {
      if (a == 0)
           devolver b
      if (b == 0)
           devolver a
      if (a <= b)
           devolver mcd(a, b mod a)
      devolver mcd(a mod b, b)
}
\end{verbatim}
\end{frame}

\begin{frame}[fragile]{Pseudoc\'odigo de soluci\'on iterativa}
\begin{verbatim}
function mcd (natural a, natural b) {
      if (a > b)
            intercambiar a y b
      // siempre mantendremos el invariante a <= b

      while (a > 0) {
            tmp = b mod a
            b = a
            a = tmp
            // notar que b mod a <= a, 
            // por eso los pusimos en ese orden
      }
      devolver b
}
\end{verbatim}
\end{frame}

\begin{frame}{Complejidad (esbozo)}

Es f\'acil ver que nuestras implementaciones son $O(min(b,c))$ pues en cada paso el n\'umero m\'as chico entre los dos involucrados decrece en al menos 1. Como cada ejecuci\'on es $O(1)$ y
paramos cuando el m\'inimo llega a 0, la ejecuci\'on completa sera $O(min(b,c))$. Despu\'es veremos que esta cota se puede mejorar mucho m\'as. \\ \bigskip

\end{frame}

\begin{frame}{Peores casos - Fibonacci}
Los peores casos del algoritmo ocurren cuando $a$ y $b$ son Fibonacci's consecutivos. Esto es as\'i pues sabemos que en cada paso se efect\'ua al menos una resta (hacer $b \mod{a}$ implica
restarle al menos $a$ a $b$) y el peor caso ocurre cuando en cada paso se efect\'ua exactamente una resta. \\ \bigskip

$mcd(F_{k}, F_{k+1}) = mcd(F_{k-1}, F_{k}) = \cdots = mcd(F_{0}, F_{1}) = mcd(0,1) = 1$ \\ \bigskip

M\'as adelante retomaremos esta observaci\'on para demostrar la complejidad real de nuestros algoritmos.
\end{frame}

\begin{frame}{Ejercicio 2.8 de la pr\'actica}
\begin{block}{Ejercicio 2.8}
\begin{enumerate}[b.]
\item Analizar el siguiente algoritmo para determinar el m\'aximo com\'un divisor entre dos n\'umeros $b$ y $c$, y
mostrar que su complejidad tambi\'en es $O(min(b,c))$. \\ \bigskip
\texttt{ \small 
$g \leftarrow min(b,c)$ \\
\htabu mientras $g > 1$ hacer \\
\htabu \tabu si $\frac{b}{g}$ y $\frac{c}{g}$ son enteros, informar $mcd = g$ y parar. \\
\htabu \tabu poner $g = g-1$ \\
\htabu informar $mcd = 1$ y parar}
\end{enumerate}
\end{block}  
\end{frame}

\begin{frame}
Notemos que el algoritmo es correcto pues recorre todos los posibles divisores de $b$ y $c$ y se queda con el mayor de ellos (que por como recorremos, es el primero
que cumple la propiedad).\\ \bigskip $mcd(b,c) \leq min(b,c)$, y recorremos a lo sumo una vez el cuerpo por cada uno de los $g$. Dicho cuerpo es $O(1)$, y se ejecuta $O(min(b,c))$
veces, por lo tanto el algoritmo completo es $O(min(b,c))$.
\end{frame}

\begin{frame}{Ejercicio 2.8 de la pr\'actica}
\begin{block}{Ejercicio 2.8}
\begin{enumerate}[c.]
\item Las complejidades calculadas para los algoritmos de las partes a. y b. son iguales. \textquestiondown Cu\'al de los dos algoritmos elegir\'a? Justificar y comentar.
\item Probar que se puede mejorar la complejidad calculada en a. demostrando que el algoritmo de Euclides
es en realidad $O(log_2(min\{b, c\}))$.
\end{enumerate}
\end{block} 
\end{frame}

\begin{frame}{Demostraci\'on de complejidad de las implementaciones del \'item a}
Recordemos que en cada paso tomamos m\'odulo del n\'umero m\'as grande sobre el m\'as chico. As\'i, en cada paso obtenemos un par de n\'umeros. Como el $mcd$ es conmutativo, ponemos primero el menor n\'umero y luego el mayor. Como $a \leq b$, en dos pasos de aplicaci\'on del lema 2 se tiene que: \\ \bigskip

$mcd(a,b) = mcd(b \mod{a}, a) = mcd(a \mod{(b \mod{a})}, b \mod{a})$ \\ \bigskip

Queremos probar que el algoritmo es $O(\lg a)$, pues $a = min(a,b)$, y para ello demostraremos que cada 2 pasos del algoritmo, $a$ se reduce a la mitad.
\end{frame}

\begin{frame}{Demostraci\'on de complejidad de las implementaciones del \'item a (cont.)}
$mcd(a,b) = mcd(b \mod{a}, a) = mcd(a \mod{(b \mod{a})}, b \mod{a})$ \\ \bigskip

Queremos probar que el algoritmo es $O(\lg a)$ (pues $a = min(a,b)$) y para ello demostraremos que cada 2 pasos del algoritmo, el m\'inimo del par actual se reduce a la mitad. Si probamos eso, como cada paso es $O(1)$ y habr\'a $O(2\lg a) = O(\lg a)$ pasos, el algoritmo ser\'a $O(\lg a)$.\\ \bigskip

\end{frame}

\begin{frame}
$mcd(a,b) = mcd(b \mod{a}, a) = mcd(a \mod{(b \mod{a})}, b \mod{a})$ \\ \bigskip

\begin{enumerate}
\item Si $b \mod{a} \leq \frac{a}{2}$ entonces el m\'inimo n\'umero del par se redujo a la mitad en tan solo un paso. Como en cada paso se reduce cada n\'umero o queda igual, luego de 2 pasos el m\'inimo n\'umero del par original se habr\'a reducido a la mitad.
\item Si $b \mod{a} > \frac{a}{2}$, veamos que $a \mod{(b \mod{a})}$ ser\'a peque\~no: \\
\tabu \tabu $a = (b \mod{a})r + s$, pues $b \mod{a} < a$ (por definici\'on de m\'odulo). \\

Ahora bien, como $b \mod{a} > \frac{a}{2}$ deducimos que $r = 1$. En otras palabras, $a = (b \mod{a}) + s \Rightarrow a - (b \mod{a}) = s < a - \frac{a}{2}$, por lo que $s < \frac{a}{2}$. 
Como $s$ es por definici\'on $a \mod{(b \mod{a})}$, acabamos de ver que el menor n\'umero del par se redujo a la mitad de su valor original en dos pasos.
\end{enumerate}

\end{frame}

\begin{frame}{Tarea}
\begin{itemize}
\item {\small \url{http://acm.timus.ru/problem.aspx?space=1\&num=1139}} \\- Un lindo problema que sale con MCD.
\end{itemize}
\end{frame}

\end{document}